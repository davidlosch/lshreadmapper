% einleitung.tex
\chapter{Einleitung}
\marginpar{Jan}
\label{sec:einleitung}

Der Zwischenbericht der Projektgruppe 583 mit dem Titel "`Algorithmen zur Entdeckung krebsauslösender Genvarianten"' präsentiert die Ergebnisse des ersten Arbeitssemesters. Wir haben uns intensiv mit dem Thema der Genomik beschäftigt, welche sich mit dem Aufbau von Genomen beschäftigt.

Alle Erbinformationen von Lebewesen befinden sich in kodierter Form in der DNA. Unterschiede in dieser sind verantwortlich dafür, dass sich Menschen unterscheiden und verschiedene spezifische Eigenschaften besitzen. Die Bestandteile der DNA können sich punktweise oder großflächig unterschieden und als Genvariante bezeichnet. Nicht jede Veränderung ist gutartig. Tritt eine Variante an einer falschen Position auf, kann sie verschiedene Krankheiten auslösen wie beispielsweise Krebs. 

Um die Erbinformationen untersuchen zu können, muss die DNA sequenziert werden. Die ersten Methoden dieser Bestimmung waren noch zeit- und kostenintensiv. Mit den \textit{Sequenzierern der nächsten Generation} wurden die Verfahren effizienter. Gleichwohl ist es auch mit ihnen nicht möglich, die gesamte DNA an einem Stück zu erfassen. Die Ausgaben der Maschinen sind kurze, unsortierte Abschnitte, welche als \textit{Reads} bezeichnet werden. Diese müssen erst durch enormen Zeitaufwand in die \textit{richtige Reihenfolge} gebracht werden, welche durch wissenschaftliche Projekte in verschiedenen Referenzgenomen festgehalten wurde.\\
Read-Mapper ordnen Reads ihre richtige Stelle im Referenzgenom zu und können sie dadurch in die richtige Reihenfolge bringen. Da Reads Genomvarianten enthalten können, ist ihre Zuordnung nicht immer eindeutig oder nur schwer zu finden. Aber auch durch Sequenzierfehler kann eine Alignierung an die richtige Position erschwert werden. Unsere Aufgabe besteht darin, einen neuen Read-Mapper zu konstruieren, welcher Reads mit Varianten korrekt an ein Referenzgenom alignieren kann, für welches bekannt ist, an welchen Positionen Varianten auftreten können. Bekannte Varianten werden also nicht mehr fälschlicherweise als Fehler betrachtet, welches noch unbekannte Varianten und Sequenzierfehler für die weitere Betrachtung hervorhebt.

Unser Ansatz verfolgt eine Kombination von verschiedenen Algorithmen. Im ersten Schritt möchten wir durch eine Form des Hashings eine Kandidatenfilterung ausführen, welche für jeden Read eine Menge von passenden Positionen zurückgibt. Im nächsten Schritt werden die Kandidaten durch einen Algorithmus, welcher auf dynamischer Programmierung basiert, an die Referenz aligniert. Durch die vorige Filterung erfolgt die Alignierung nur an kurzen Stellen der Referenz, ist das semi-globale Alignment erst akzeptabel zu verwenden. Dieser Ansatz wird ausführlich in den Kapiteln \ref{sec:lsh} und \ref{sec:align}.

Desweiteren unterliegt unser Bericht folgender Gliederung: In den einführenden Kapiteln \ref{sec:orga} und ~\ref{sec:bio} werden grundlegende Informationen und Basiswissen vermittelt, welche zum Verständnis des weiteren Textes unerlässlich sind. Dabei handelt es sich einerseits um die organisatorische Struktur unserer Projektgruppe und die verwendeten technischen Hilfsmittel, andererseits um biologische Grundlagen über den Aufbau und die Sequenzierung von DNA. Dem folgen in Kapitel \ref{sec:stats} statistische Auswertungen über das Vorkommen von bereits bekannten Varianten im Humangenom.

Dem schließt sich, beginnend mit Kapitel \ref{sec:data}, der technische Teil an. Zunächst werden verschiedene Dateiformate beschrieben, welche von unseren Algorithmen und Datenstrukturen verwendet werden. Dem folgen die bereits angesprochenen Kapitel über unseren Ansatz der kombinierten Algorithmen. 

Zum Abschluss des Berichts wird ein Fazit der ersten Hälfte der Projektdauer gezogen und erste Ergebnisse präsentiert. Außerdem wird ein Ausblick über die Aufgaben und Ziele des zweiten Abschnitts gegeben. 