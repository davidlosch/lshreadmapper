% einleitung.tex
\chapter{Einleitung}
\marginpar{Jan}
\label{sec:einleitung}

Der Zwischenbericht der Projektgruppe 583 mit dem Titel "`Algorithmen zur Entdeckung krebsauslösender Genvarianten"' präsentiert die Ergebnisse des ersten Arbeitssemesters. Wir haben uns intensiv mit dem Thema der Genomik beschäftigt, welche sich mit dem Aufbau von Genomen befasst.

Alle Erbinformationen von Lebewesen befinden sich in kodierter Form in der DNA. Unterschiede in dieser sind verantwortlich dafür, dass sich Menschen unterscheiden und verschiedene spezifische Eigenschaften besitzen. Die Bestandteile der DNA können sich punktweise oder großflächig unterschieden, was als Genvariante bezeichnet wird. Jedoch sind nicht alle diese Varianten gutartig. Tritt eine Veränderung an einer falschen Position auf, kann sie Auslöser für verschiedene Krankheiten wie beispielsweise Krebs sein.

Um die Erbinformationen untersuchen zu können, muss die DNA sequenziert werden. Die ersten Sequenzierungsmethoden waren noch zeit- und kostenintensiv. Mit den \textit{Sequenzierern der nächsten Generation} wurden die Verfahren effizienter. Gleichwohl ist es auch mit ihnen nicht möglich, die gesamte DNA an einem Stück zu erfassen. Die Ausgaben der Maschinen sind kurze, unsortierte Abschnitte, welche als \textit{Reads} bezeichnet werden. Diese müssen erst mit großen Rechenaufwand in die \textit{richtige Reihenfolge} gebracht werden, um das Genom rekonstruieren zu können. Wissenschaftliche Projekte wie das \citet{1000Genomes} haben diese Arbeit in verschiedenen Referenzgenomen festgehalten.

Readmapper ordnen Reads ihre richtige Stelle im Referenzgenom zu und können sie dadurch in die richtige Reihenfolge bringen. Da Reads Genomvarianten enthalten können, ist ihre Zuordnung nicht immer eindeutig oder nur schwer zu finden. Aber auch durch Sequenzierfehler kann eine Alignierung an die richtige Position erschwert werden. Unsere Aufgabe besteht darin, einen neuen Readmapper zu konstruieren, welcher Reads mit Varianten korrekt an ein Referenzgenom alignieren kann, für welches vermerkt wurde, an welchen Positionen bereits bekannte Varianten auftreten können. Bekannte Varianten werden dabei nicht mehr fälschlicherweise als Abweichung von Referenzgenomen erkannt. Hierdurch werden unbekannte Varianten und Sequenzierfehler besser hervorgehoben, was deren Nachweis in nachfolgenden Verarbeitungsschritten erleichtert.

Unser Ansatz verfolgt ein zweistufiges Verfahren. Im ersten Schritt führen wir mittels eines Hashingverfahrens ein Mapping durch, welches für jeden Read eine Menge von passenden Positionen erzeugt. Im nächsten Schritt werden die Reads durch einen auf dynamischer Programmierung basierenden Algorithmus an die Referenz aligniert. Durch das vorherige Mapping erfolgt die Alignierung nur an kurze Referenzabschnitte, wodurch der Alignierungsalgorithmus überhaupt erst akzeptable Laufzeiten erreichen kann. Die Algorithmen werden ausführlich in den Kapiteln \ref{sec:lsh} und \ref{sec:align} beschrieben.

Des Weiteren unterliegt unser Bericht folgender Gliederung: In den einführenden Kapiteln \ref{sec:orga} und ~\ref{sec:bio} werden grundlegende Informationen zur Projektgruppe sowie Basiswissen vermittelt. Dabei handelt es sich einerseits um die organisatorische Struktur unserer Projektgruppe und die verwendeten technischen Hilfsmittel, andererseits um biologische Grundlagen über den Aufbau und die Sequenzierung von DNA. In Kapitel \ref{sec:stats} folgen statistische Auswertungen über das Vorkommen von bereits bekannten Varianten im Humangenom.

Dem schließt sich, beginnend mit Kapitel \ref{sec:data}, der technische Teil an. Zunächst werden verschiedene Dateiformate beschrieben, welche von unseren Algorithmen und Datenstrukturen verwendet werden. Darauf folgen die bereits angesprochenen Kapitel über die von uns verwendeten Algorithmen.

Zum Abschluss des Berichts wird ein Fazit der ersten Hälfte der Projektdauer gezogen und erste Ergebnisse präsentiert. Außerdem wird ein Ausblick über die Aufgaben und Ziele des zweiten Abschnitts gegeben. 